\documentclass[10pt]{report}
\usepackage{fontspec}
\usepackage[onehalfspacing]{setspace}
\usepackage{lscape} 
\usepackage{graphicx}
\usepackage[nswissgerman]{babel}
\usepackage[german = swiss]{csquotes} 
\usepackage{float}
\usepackage{enumitem}
\setlist{nolistsep}

% In diesem Blcok können die Angaben zur Arbeit erfasst werden, welche dann mit dem jeweiligen Befehl im Rest des Dokumentes verwendet werden können. 
\newcommand{\titel}{$Titel der Arbeit}
\newcommand{\desc}{$Name des Leistungsnachweis}
\newcommand{\authors}{$Autoren}
\newcommand{\lecturer}{$Betreuer}
\newcommand{\module}{$Modul oder $Kurs}
\newcommand{\program}{MScBA}
\newcommand{\major}{Information and Data Management}
\newcommand{\semester}{Frühlingsemester 2022}


\usepackage{fancyhdr}
\pagestyle{fancyplain}
\fancyhf{}
\lhead{\footnotesize \titel}
\chead{}
\rhead{}
\lfoot{\footnotesize \authors}
\cfoot{}
\rfoot{\footnotesize \thepage}

\usepackage[a4paper, left=3cm, right=2.5cm, top=2.5cm, bottom=2cm]{geometry}
\usepackage{times}
\linespread{1.3}
 
\usepackage[
    backend=biber,
    style=apa,
  ]{biblatex}
 \renewcommand{\arraystretch}{1.7}
 \addbibresource{referenzen.bib}

\begin{document}
\begin{titlepage}
\begin{center}
{\scshape\LARGE \titel \par}
\vspace{1.5cm}
    \LARGE
	\textbf{\desc}
\end{center}
\begin{center}
	\vspace{5cm}
	\large
	\begin{tabular}{l l}
    Modul: & \module \\
    Autoren: & \authors\\
    Referent: & \lecturer \\
    Datum: & \today \\
  \end{tabular}
\end{center}

\end{titlepage}

\newpage 

\tableofcontents
\listoffigures
\listoftables

\newpage
\chapter{Einleitung}

Hello World \cite{LUNDY20151057} Hier kommt der Text

\chapter{Projektplan}


\chapter{Hauptteil}

\chapter{Schluss}

\newpage
\appendix
\chapter{An Introduction to Lua\TeX}
\section{Nochwas}
\section{Irgendwas}

\newpage
\printbibliography

\end{document}
